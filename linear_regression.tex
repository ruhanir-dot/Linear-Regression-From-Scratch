\documentclass{article}

% Language setting
% Replace `english' with e.g. `spanish' to change the document language

% Set page size and margins
% Replace `letterpaper' with `a4paper' for UK/EU standard size
\usepackage[letterpaper,top=2cm,bottom=2cm,left=3cm,right=3cm,marginparwidth=1.75cm]{geometry}

% Useful packages
\usepackage{amsmath}
\usepackage{graphicx}
\usepackage[colorlinks=true, allcolors=blue]{hyperref}

\title{Linear Regression From Scratch}
\author{Ruhani Rekhi}

\begin{document}
\maketitle


\section{Introduction}

The goal of this project is to implement linear regression from scratch using Python and NumPy to understand the underlying mathematics and algorithms behind linear regression.
The importance of the project lies in its ability to provide a foundational understanding of linear regression as well as insights into how the algorithm works, including the concepts of cost functions, gradient descent, and model evaluation.


\subsection{Basics Of Linear Regression}
Regression is a method used to model the relationship between a dependent variable and one or multiple independent variables. Linear regression is a specific type of regression analysis where we model the relationsip between the dependent variable and independent variable(s) as a linear function.
The main goal of linear regression is to determine the function of a best fit line that minimizes error. Error in this case is defined as difference between the predicted and actual value. Mathematically written as $ e = y - \hat{y}  $ where $\hat{y}$ is the predicted value. Our function is defined as $ \hat{y} = b_0 + b_1x_1 $ where $b_0$ is our y-intercept and $b_1$ is the slope. 

\subsection{Minimizing the Error Function}

To begin minimizing the error function we need to define what the error function is. 

Let $E$ be our error function defined as :
\[E = \frac{1}{n} \sum_{i}^{n}(y_i - \hat{y_i})^2
    = \frac{1}{n} \sum_{i}^{n}(y_i - (b_0 + b_1x_1))^2
\]


\noindent This value is referred to as the MSE or mean squared error. In layman terms, we are taking the difference from all the actual points y-value
and what our linear function would predict what the y-value would be and square the difference and divide it by the total number of points to 
derive our MSE. Now that we have defined our error function $E$ we now want to minimize our E : find the function with lowest possible $E$.
\end{document}